%!TEX TS-program = xelatex
\documentclass{nihbiosketch}
\usepackage{indentfirst}
% \usepackage{draftwatermark}  % delete this in your document!
% \SetWatermarkText{Sample}    % delete this in your document!
% \SetWatermarkLightness{0.9}  % delete this in your document!

%------------------------------------------------------------------------------

\name{Goff, Loyal Andrew}
\eracommons{LAGOFF}
\position{Assistant Professor of Neuroscience}

\begin{document}
%------------------------------------------------------------------------------

\begin{education}
The College of New Jersey, Ewing, NJ  			      & B.S           & 05/2001  & Biology \\
Rutgers University, Piscataway, NJ              	  & Ph.D.         & 04/2008  & Cell \& Developmental Biology \\
Massachusetts Institute of Technology, Cambridge, MA  & Postdoctoral  & 01/2014  & Computational Biology \\
Harvard University, Cambridge, MA  					  & Postdoctoral  & 08/2014  & ncRNA Biology / Neurodevelopment \\
\end{education}

\section{Personal Statement}
	My research group seeks to characterize neural cell type specification, key cell fate decisions during development, and the effects of disease-associated mutations on these choices. My focus is on the characterization of cell-type-specific trajectories in developmental processes within the central nervous system, context-dependent (cell type) modulation of gene expression in response to sensory manipulation, and the reconstruction of continuous biological processes in neurodegenerative and developmental disorders. 
	In teaching, as in research, I emphasize collaboration. Cooperative learning between peers provides an opportunity to discover novel concepts and viewpoints and fosters a collegial environment. I encourage inquiry-based learning in both wet lab as well as computational facets of training, and expect trainees to be able to identify, propose solutions for, and address important outstanding questions in biology that are of particular interest to them. I am committed to, and strongly support the need for scientists from diverse backgrounds and continue to be proudly associated with the PREP program at Johns Hopkins University.  As part of this program, I am acting as a co-mentor for one PREP scholar, Alejandra Patino. Since joining our lab, Ale has designed and implemented a novel experimental study at the intersection of molecular biology, neuroscience, and autism research and has developed cutting-edge single cell workflows to identify and describe two key cellular populations in the mouse hypothalamus involved in a parallel processing of social behavioral deficits observed in autism. 
	
%This section is for papers associated with the Personal Statement
% \begin{enumerate}
% \item Clark B, Stein-O’Brien G, Shiau F, Cannon GH, Davis E, Sherman T, Rajaii F, James-Esposito R, Gronostajski R, Fertig EJ, \textbf{Goff LA*}, Blackshaw S*. Comprehensive analysis of retinal development at single cell resolution identifies NFI factors as essential for mitotic exit and specification of late-born cells. BioRxiv 378950 [Preprint]. July 30, 2018 [cited 2018 Sept 21]. Available from: \url{https://doi.org/10.1101/378950} *Co-corresponding authors

% \item Stein-O’Brien GL, Clark BS, Sherman T, Zibetti C, Hu Q, Sealfon R, Liu S, Qian J, Colantuoni C, Blackshaw S, \textbf{Goff LA*}, Fertig EJ*. Decomposing cell identity for transfer learning across cellular measurements, platforms, tissues, and species. BioRxiv 395004 [Preprint]. August 20, 2018 [cited 2018 Sept 21]. Available from: \url{https://doi.org/10.1101/395004} *Co-corresponding authors

% \end{enumerate}

%------------------------------------------------------------------------------
\section{Positions and Honors}

\subsection*{Positions and Employment}
\begin{datetbl}
2008--2008	& Research Assistant, Rutgers University -- Cell Biology and Neuroscience \\
2008--2014	& Postdoctoral Fellow, MIT -- Computer Science and Artificial Intelligence Lab \\
2009--2014	& Postdoctoral Fellow, Harvard University -- Stem Cells and Regenerative Biology \\
2014--		& Assistant Professor, Johns Hopkins School of Medicine -- Institute of Genetic Medicine \\
2015--		& Associate Member, Kavli Neurodiscovery Institute -- Johns Hopkins University SOM \\
\end{datetbl}

% \subsection*{Other Experience and Professional Memberships}
% \begin{datetbl}
% 2007--		& Member, Society for Neuroscience \\
% 2015--           & Member, American Society for Human Genetics \\
% \end{datetbl}

% \subsection*{Honors}
% \begin{datetbl}
% 2009            & NSF Postdoctoral Research Fellowship in Biology: Biological Informatics program \\
% 2009            & NIH Ruth L. Kirschstein NRSA for Individual Postdoctoral Fellows (awarded but declined) \\
% 2018	        & Johns Hopkins University Catalyst Award \\
% \end{datetbl}

%------------------------------------------------------------------------------

\section{Contribution to Science}

\begin{enumerate}

% \item I have a standing interest in the various mechanisms of RNA-mediated biological activity and regulation, and the potential roles for this class of macromolecule in the acquisition and maintenance of specific cellular identities. 
% \begin{enumerate}

% \item	\textbf{Goff, L.A.}, Yang, M., Bowers, J., Getts, R.C., Padgett, R.W., and Hart, R.P. (2005). Rational probe optimization and enhanced detection strategy for microRNAs using microarrays. RNA Biol 2, 93–100.

% \item	\textbf{Goff, L.A.}, Boucher, S., Ricupero, C.L., Fenstermacher, S., Swerdel, M., Chase, L.G., Adams, C.C., Chesnut, J., Lakshmipathy, U., and Hart, R.P. (2008). Differentiating human multipotent mesenchymal stromal cells regulate microRNAs: prediction of microRNA regulation by PDGF during osteogenesis. Exp. Hematol. 36, 1354–1369. PMCID: PMC2782644

% \item	\textbf{Goff, L.A.*}, Davila, J.*, Swerdel, M.R., Moore, J.C., Cohen, R.I., Wu, H., Sun, Y.E., and Hart, R.P. (2009). Ago2 immunoprecipitation identifies predicted microRNAs in human embryonic stem cells and neural precursors. PLoS ONE 4, e7192. *Authors contributed equally. PMCID: PMC2745660

% \item	Davila, J.L.*, \textbf{Goff, L.A.*}, Ricupero, C.L.*, Camarillo, C., Oni, E.N., Swerdel, M.R., Toro-Ramos, A.J., Li, J., and Hart, R.P. (2014). A Positive Feedback Mechanism That Regulates Expression of miR-9 during Neurogenesis. PLoS ONE 9, e94348. *Authors contributed equally. PMCID: PMC3979806


% \end{enumerate}


\item Long non-coding RNAs (lncRNAs) are regulatory RNA genes with a high degree of cell-type specificity that are likely to contribute to specific cellular identities and functions. I was responsible for the identification of thousands of human and mouse lncRNAs that are induced during cellular differentiation programs or restricted to subpopulations of cells. My research has determined that indeed several lncRNA loci are required for proper development of the brain.     

\begin{enumerate}

\item	 Cabili, M.N., Trapnell, C., \textbf{Goff, L.}, Koziol, M., Tazon-Vega, B., Regev, A., and Rinn, J.L. (2011). Integrative annotation of human large intergenic noncoding RNAs reveals global properties and specific subclasses. Genes \& Development. PMCID: PMC3185964

\item	Sun, L.*, \textbf{Goff, L.A.*}, Trapnell, C.*, Alexander, R., Lo, K.A., Hacisuleyman, E., Sauvageau, M., Tazon-Vega, B., Kelley, D.R., Hendrickson, D.G., et al. (2013). Long noncoding RNAs regulate adipogenesis. Proceedings of the National Academy of Sciences 110, 3387–3392. *Authors contributed equally. PMCID: PMC3587215

\item	Sauvageau, M.*, \textbf{Goff, L.A.*}, Lodato, S.*, Bonev, B., Groff, A.F., Gerhardinger, C., Sanchez-Gomez, D.B., Hacisuleyman, E., Li, E., Spence, M., et al. (2013). Multiple knockout mouse models reveal lincRNAs are required for life and brain development. Elife 2, e01749. *Authors contributed equally. PMCID: PMC3874104

\item	\textbf{Goff, L.A.*}, Groff, A.F.*, Sauvageau, M.*, Trayes-Gibson, Z., Sanchez-Gomez, D.b., Morse, M., Martin, R.D., Elcavage, L.E., Liapis, S.C., Gonzalez-Celeiro, M., Plana, Ol., Li, E., Gerhardinger, C., Tomassay, G.S., Arlotta, P., Rinn, J.L., (2015) Spatiotemporal expression and transcriptional perturbations by long noncoding RNAs in the mouse brain. PNAS 112(22): 6855-6862. * Authors Contributed Equally. PMCID: PMC4460505

\end{enumerate}

\item I have been involved in the development of several key computational/informatic tools to facilitate increased adoption of, and access to RNA-Seq data. I am responsible for the development of several training programs aimed at encouraging molecular biologist to learn to analyze these complex data themselves.

\begin{enumerate}   

\item	Trapnell, C., Roberts, A., \textbf{Goff, L.}, Pertea, G., Kim, D., Kelley, D.R., Pimentel, H., Salzberg, S.L., Rinn, J.L., and Pachter, L. (2012). Differential gene and transcript expression analysis of RNA-seq experiments with TopHat and Cufflinks. Nat Protoc 7, 562–578. PMCID: PMC3334321

%\item	Washietl, S., Will, S., Hendrix, D.A., \textbf{Goff, L.A.}, Rinn, J.L., Berger, B., and Kellis, M. (2012). Computational analysis of noncoding RNAs. Wiley Interdiscip Rev RNA 3, 759–778. PMCID: PMC3472101

\item	Trapnell, C., Hendrickson, D.G., Sauvageau, M., \textbf{Goff, L.}, Rinn, J.L., and Pachter, L. (2013). Differential analysis of gene regulation at transcript resolution with RNA-seq. Nat. Biotechnol. 31, 46–53. PMCID: PMC3869392

\item	Stein-O’Brien GL, Arora R, Culhane AC, Favorov A, Greene C, \textbf{Goff LA}, Li Y, Ngom A, Ochs MF, Xu Y, Fertig EJ. (2018) Enter the matrix: Factorization Uncovers Knowledge from Omics. Trends In Genetics. 34(10), 790-805. 

\item Stein-O’Brien GL, Clark BS, Sherman T, Zibetti C, Hu Q, Sealfon R, Liu S, Qian J, Colantuoni C, Blackshaw S, Goff LA*, Fertig EJ*. Decomposing cell identity for transfer learning across cellular measurements, platforms, tissues, and species. BioRxiv 395004 [Preprint]. August 20, 2018 [cited 2018 Sept 21]. Available from: \url{https://doi.org/10.1101/395004} *Co-corresponding authors

\end{enumerate}

\item I have contributed to the identification of novel mechanisms for RNA-mediated regulation and to the development of transcriptome-wide technologies to elucidate functional RNA elements.

\begin{enumerate}
\item	Di Ruscio, A., Ebralidze, A.K., Benoukraf, T., Amabile, G., \textbf{Goff, L.A.}, Terragni, J., Figueroa, M.E., De Figueiredo Pontes, L.L., Alberich-Jorda, M., Zhang, P., et al. (2013). DNMT1-interacting RNAs block gene-specific DNA methylation. Nature 503, 371–376. PMCID: PMC3870304

\item	Gregory, B.D., Rinn, J., Li, F., Trapnell, C., and \textbf{Goff, L.A.} (2015). High-throughput methodology for identifying RNA-protein interactions transcriptome-wide. US Patent Office. US9097708 B2.

\item	Silverman, I.M., Li, F., Alexander, A., \textbf{Goff, L.,} Trapnell, C., Rinn, J.L., and Gregory, B.D. (2014). RNase-mediated protein footprint sequencing reveals protein-binding sites throughout the human transcriptome. Genome Biol 15, R3. PMCID: PMC4053792

\item	Hacisuleyman, E.*, \textbf{Goff, L.A.*}, Trapnell, C., Williams, A., Henao-Mejia, J., Sun, L., McClanahan, P., Hendrickson, D.G., Sauvageau, M., Kelley, D.R., et al. (2014). Topological organization of multichromosomal regions by the long intergenic noncoding RNA Firre. Nat. Struct. Mol. Biol. 21, 198–206. *Authors contributed equally. PMCID: PMC3950333

\end{enumerate}

\item Recent work has focused on transcriptional characterization of subpopulations of neuronal cell types during development using single cell RNA-Seq analysis. Beyond simply classifying enriched cell types, recent work has delved deeper to identify and define the sources of variation (basis vectors) that independently contribute to the transcriptional state of a cell.

\begin{enumerate}
\item	Molyneaux, B.J.*, \textbf{Goff, L.A.*}, Brettler, A.C., Chen, H.-H., Brown, J.R., Hrvatin, S., Rinn, J.L., and Arlotta, P. (2015). DeCoN: genome-wide analysis of in vivo transcriptional dynamics during pyramidal neuron fate selection in neocortex. Neuron 85, 275–288. *Authors Contributed Equally PMCID: PMC4430475

\item	Hook P, McClymont SA, Cannon GH, Law WD, Morton JA, \textbf{Goff LA*}, McCallion AS* (2017) Single-cell RNA-seq of dopaminergic neurons informs candidate gene selection for sporadic Parkinson's disease. Am J Hum Genet. 2018 Mar 1;102(3):427-446. doi: 10.1016/j.ajhg.2018.02.001. *Co-corresponding authors

\item	Cheveé M, Robertson JJ, Cannon GH, Brown SP, \textbf{Goff LA }(2018) Variation in activity state, axonal projection and position define the transcriptional identity of individual neocortical projection neurons. Cell Reports. Jan9; 22(2):441-455. PMID: 29320739

\item Clark B, Stein-O’Brien G, Shiau F, Cannon GH, Davis E, Sherman T, Rajaii F, James-Esposito R, Gronostajski R, Fertig EJ, \textbf{Goff LA*}, Blackshaw S*. Comprehensive analysis of retinal development at single cell resolution identifies NFI factors as essential for mitotic exit and specification of late-born cells. BioRxiv 378950 [Preprint]. July 30, 2018 [cited 2018 Sept 21]. Available from: \url{https://doi.org/10.1101/378950} *Co-corresponding authors

\end{enumerate}

\end{enumerate}

\subsection*{Complete List of Published Work in MyBibliography:} 
\url{https://www.ncbi.nlm.nih.gov/myncbi/browse/collection/40255272/?sort=date&direction=descending}

%------------------------------------------------------------------------------

\section{Research Support}

\subsection*{Ongoing Research Support}

\grantinfo{2016-MSCRFI-2805}{Goff}{06/01/16--05/31/19}
{Single cell analysis of hippocampal neurogenesis defects in Kabuki Syndrome 1}
%{The goal of this project is to characterize the molecular mechanisms responsible for learning and memory dysfunction as a result of defective neurogenesis associated with Kabuki Syndrome 1 patient mutations.}
{Role: PI}

%\bigskip

\grantinfo{IOS-1665692activity NSF}{Brown/Goff}{03/01/17--02/28/21}
{Cell type specific gene expression differences induced by experience-dependent plasticity}
%{This proposal aims to examine the transcriptional profiles of distinct neuronal types within mouse sensory cortex to identify common and cell type-specific molecular changes induced by well-established paradigms of experience-dependent plasticity.}
{Role: Co-PI}

%\bigskip

\grantinfo{TargetALS Foundation}{Goff}{05/01/17--4/30/19}
{Cellular Mechanisms of Cortical Hyperexcitability}
%{This project will explore the cell-type-specific effects of familial ALS mutations on hyperexcitability of cortical neurons, and the common and distinct gene expression changes that evoke this phenotype in ALS mouse models.}
{Role: PI}

%\bigskip

\grantinfo{Chan Zuckerberg Institute Award}{Goff}{04/01/18--3/31/19}
{Rapid exploration, interpretation, and comparison of discrete basis vectors contributing to transcriptional Signatures of single cells at the scale of the Human Cell Atlas with ProjectoR}
%{The major goals of this project are to develop computational tools and workflows for transfer learning methods in single cell analysis.}
{Role: PI}

%\bigskip

\grantinfo{1R21AI139358-01 NIAID}{Potter/McMeniman/Goff}{05/01/18--4/30/2020}
{Identification and characterization of mosquito sensory neurons detecting human-related cues }
%{The major goals of this project are to 1) To develop a genetic method in Aedes aegypti mosquitoes for labeling sensory neurons activated by human-related odorants and 2) To identify candidate molecular receptors from activated mosquito sensory neurons that may be targeted for novel mosquito behavioral disruption strategies.}
{Role: Co-Investigator}

%\bigskip

\grantinfo{Helis Foundation Award}{Goff/McCallion}{10/01/18--9/31/21}
{Single Cell RNA-Seq description of Dopamine Neuron Subtypes and Their Responses to Familial Parkinson’s Mutations}
%{The major goals of this project are to determine the specific pathways and molecular mechanisms disrupted by Parkinson disease (PD) in each cellular subtype to determine how different dopaminergic neuron subtypes are affected by a single disease mutation and explore novel ways to improve PD patient care.}
{Role: Co-PI}

%\bigskip

%------------------------------------------------------------------------------

% \subsection*{Completed Research Support}

% \grantinfo{JHU Science of Learning Institute}{Brown/Goff}{06/01/16--05/31/18}
% {Cell-type specific heterogeneity in experience-induced gene expression}
% {The major goals of this project are to generate preliminary data on common and variable transcriptional signatures of plasticity across distinct populations of neuronal cell types and sensory inputs.}
% {Role: Co-PI}

% \bigskip

% \grantinfo{JHU Synergy Award}{Goff/Fertig}{07/01/17--6/30/18}
% {Systematic characterization of transcriptional variation in retinal development at single cell resolution}
% {The major goals of this project are to establish the transcriptional landscape of the developing mouse retina to identify key factors governing fate specification and changes in progenitor cell competency.}
% {Role: Co-PI}

% \bigskip
\end{document}
